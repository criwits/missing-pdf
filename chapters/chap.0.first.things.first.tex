\setcounter{chapter}{-1}

\chapter{一些约定与预备知识}

尽管《Missing》自认为受众是「电脑小白」,但我们依然不得不要求读者有一些最最基本的知识和操作经验。
此外,为了便于后续的讲述,我们会进行一些约定。

\section{容量与容量单位}

我们约定存储在电脑中的数据的容量单位「GB」「MB」「KB」等的关系如下:
\[1\,\mathrm{TB}=1024\,\mathrm{GB}=1024\times1024\,\mathrm{MB}=1024^3\,\mathrm{KB}=1024^4\,\text{字节}\]
即,我们统一使用「1024 进位」而不是「1000 进位」作为单位换算的系数。
此外,我们有时会略去这些单位最后的字母 B,即文中可能用「1 T」来表示「1 TB」。

你可能会问为什么要特别说明这里是「1024 进位」而不是「1000 进位」。
如果你有买过 U 盘,你会发现标称「128 GB」的 U 盘实际可用的容量只有 119 GB 左右。
这是因为生产 U 盘的厂家是用「1000 进位」来计算容量,而电脑自身是用「1024 进位」来计算容量的。
在这里我们进行容量单位的约定,是为了避免这种争议。

\section{文中的标记符号}

在《Missing》中,我们使用方头括号「【】」来标记所有屏幕上字面显示的选项。
例如,当我们希望你右键桌面上的图标
\begin{figure}[htb!]
  \centering
  \includegraphics[width=2cm]{assets/This_PC.png}
\end{figure}

\noindent 时,我们会称「右键【此电脑】」。

我们使用右箭头「→」来表示下一步操作。
例如,「右键【此电脑】→【属性】」的意思是,右键桌面上的【此电脑】图标,然后在弹出的菜单中点击【属性】。

\section{快捷键的操作说明}

如果你按快捷键(多个键盘按键的组合键)后,电脑并没有行使理想中的功能,可能是你的按法不对。
快捷键的按法并不是「同时按下所有的键」,而是「依次序按下各键不松手,最后一起松开」。
例如,若要按快捷键「\keys{ + Shift + S}」:

\begin{itemize}
  \item 先按住 \keys{} 键(\keys{} 键上印有Window 徽标「」,一般来说这个键在 \keys{Ctrl} 和 \keys{Alt} 之间)不要松手;
  \item 再按住 \keys{Shift} 键,同样不要松手;
  \item 接着按一下 \keys{S} 键,然后松开全部按键。
\end{itemize}

\section{\keys{F1} -- \keys{F12} 功能键的使用说明}

对于笔记本电脑,其键盘最上方一排按键(\keys{F1} -- \keys{F12} 功能键)往往具有两重功能:
「它们本身的功能」和「它们的拓展功能」。

所谓「它们本身的功能」,指的就是 app 中规定的这些键的功能。
例如,在浏览器中,\keys{F5} 通常用来刷新页面,那么 \keys{F5} 键的「本身的功能」在浏览器中就是刷新页面。

所谓「它们的拓展功能」,指的是这些键上面画的图标所规定的额外功能。
例如,笔者的笔记本中 \keys{F5} 键上画有亮度降低的符号,因此 \keys{F5} 键的「拓展功能」就是降低屏幕亮度。

利用键盘左下角的 \keys{Fn} 键可以在这两种功能中切换。
具体地,对于一台电脑具有 \keys{Fn} 键设计的电脑,它可能是下列两种情况中的一种:

\begin{itemize}
  \item 直接按 \keys{F1} -- \keys{F12} 功能键可以行使它们本来的功能,按住 \keys{Fn} 的同时再按 \keys{F1} -- \keys{F12} 则行使它们的拓展功能。\\
    例如:如果 \keys{F5} 功能如上,那么,在这种情况下,按 \keys{F5} 可以在浏览器中刷新页面,按 \keys{Fn + F5} 可以降低屏幕亮度。
  \item 直接按 \keys{F1} -- \keys{F12} 功能键可以行使它们的拓展功能,按住 \keys{Fn} 的同时再按 \keys{F1} -- \keys{F12} 则行使它们本来的功能。\\
    例如:如果 \keys{F5} 功能如上,那么,在这种情况下,按 \keys{F5} 可以降低屏幕亮度,按 \keys{Fn + F5} 可以在浏览器中刷新页面。
\end{itemize}

你可以通过打开浏览器,打开某个页面,然后按 \keys{F5} 来测试你的电脑属于上面两种情况中的哪一种。
更一般地,用其他带有 \keys{F1} -- \keys{F12} 快捷键的软件来测试也是可以的。

很多笔记本电脑提供了一种快捷的方法在这两种模式中切换。
对于一些品牌的笔记本,你可以通过按 \keys{Fn + Esc} 来切换这两种模式。
另一些笔记本可能需要使用专门的软件(例如,Lenovo Vantage)或进入 BIOS 设置。
具体请查阅你设备的操作指南或自行上网搜索。

\section{有关「重启」的说明}

由于自 Windows 8 以来的 Windows 系统引入了「快速启动」的机制,现在「重启」这个过程并不等价于「先关机再开机」的过程。

故,若在文中提及「重启」这样的操作,请一定是选择开始菜单中的「重启」选项,而非点选「关机」后再手动打开电脑。

\practice

\begin{enumerate}
  \item 计算 1 GB 等于多少 KB?等于多少字节?假设一个汉字占两个字节,1 GB 大约可以记录多少个汉字?
  \item 尝试计算,一只按「1000 进位」计算得到容量为 64 GB 的 U 盘,它按「1024 进位」得到的容量是多少?
  \item 在自己电脑上尝试这些快捷键:
    \begin{enumerate}
      \item \keys{ + Shift + S} (仅限 Windows 10 / 11)
      \item \keys{Ctrl + Shift + Esc}
      \item \keys{ + D}
    \end{enumerate}
  \item 了解并试验自己笔记本电脑的 \keys{F1} -- \keys{F12} 功能键的拓展功能。
\end{enumerate}