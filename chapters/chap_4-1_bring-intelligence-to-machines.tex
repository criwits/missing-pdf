\chapter{从加减乘除到 AI 对话}
\label{cha:bring-intelligence-to-machines}

\begin{intro}
  这一章,我们将介绍超越篇的第一个领域——人工智能,或简称「AI」(Artificial Intelligence)。AI 技术让电脑不再是冷冰冰的机器,而拥有了一定模仿人类思维的能力。今天,诸如 AI 对话、AI 绘画、AI 写作等技术正在为人们的生产生活提供别样的帮助。阅读完这一章,你或许能找到下面这些问题的答案:
  \begin{itemize}
    \item AI 在我们生活中真的无处不在吗?
    \item 我想体验一下现在最新的 AI 技术!
    \item 到底是什么让 AI 能够模仿人类的思维,实现「智能」?
    \item 未来,AI 技术会如何发展?我们需要担心吗?我们应该怎么办?
  \end{itemize}
\end{intro}

\section{让机器像人一样思考}

\subsection{人工实现的「智能」,无处不在的 AI}

「人工智能」一词,拆开来看,由「人工」和「智能」组成。「智能」是人类的专利,我们的意识、思考、判断和情感等都是智能的表现。「人工」修饰「智能」,昭示这种智能由人工实现,让机器也具有识别、决策、判断,甚至分析、学习和创造的能力。

几千年来,无论是东方还是西方,人们都不约而同地畅想过各式各样的「自动人偶」,这也许是人工智能最早的概念先驱。近代,随着数学和逻辑学的发展,学者们开始研究如何把世间一切逻辑都用数学语言表达出来,真正的人工智能在此奠下基础。到了 20 世纪,电子计算机诞生,人们也开始使用它模拟人类的思维。随后短短几十年间,人工智能技术经历了几次高潮和低谷,而当下,恰好处于人工智能技术发展的一个新高峰。

不知不觉间,人工智能技术其实已经完完全全融入了我们生活的每一个角落。当我们「刷脸」步入小区大门时,人脸识别系统能像门卫一样辨认我们的身份;当我们上网购物时,智能客服像真人一样回答着我们的问题。手机里,操作系统如同私人助理,根据我们使用 app 的习惯调整性能;工厂中,机器人仿佛勤劳的工人,正在代替我们完成重复性劳动……这些我们早已认为理所当然、司空见惯的技术,其实都是人工智能技术的应用——计算机视觉、自然语言处理、推荐算法、智能机器人……\regcolor{在这些领域,人工智能已经有了成熟的应用,称为「传统 AI」。}

近年来,「AI 绘画」「AI 聊天」等技术再一次高调地让「AI」一词进入人们的视野。这些技术看似各不相同,但它们有着一个共同的特点:都是由 AI 来生成内容。这些 AI 生成的画作、文本和音频等的质量和效果与人类相差无几,甚至有时比人类的作品更令人惊艳。\regcolor{这种技术称为「生成式人工智能」,简称「生成式 AI」,通常缩写成英文 AIGC(Artificial Intelligence Generated Content)。}如果说用于传统 AI 技术是让 AI 去干活,那么 AIGC 技术则是让 AI 去创作。

AIGC 的诞生可谓一座里程碑。在 AIGC 技术铺开之前,人们通常认为「创作」是人类思维独有的能力,AI 技术只能是让机器代替人类做一些简单而枯燥的工作。然而,AIGC 技术的出现撼动了这种观念。自此之后,人们对 AI 技术的未来充满了无限的憧憬,也让人们对 AI 技术的潜在风险产生了更多的担忧。

\subsection{模型、训练和推理}

在 AI 领域中,\regcolor{「模型」(model)指的是用于完成特定任务的结构,可以简单理解为一个「程序」}。例如,用于 AI 绘画的模型,输入你的文字描述,就能输出一幅符合描述的图画。而用于 AI 对话的模型,输入你的问题,就能输出一个符合问题的回答。\regcolor{模型通过你的输入返回给你一些输出,这个过程称为模型的「推理」。}

\begin{figure}[htb!]
  \centering
  \includegraphics[width=.65\textwidth]{assets/surpass/AI_models.pdf}
  \caption{不同的 AI 模型}
  \label{fig:AI_models}
\end{figure}

模型的内部是复杂的数学运算,其中的大量参数并非人们直接设定,而是机器自行「学习」得来。\regcolor{这个学习的过程称为「训练」},它需要大量的数据和计算资源。例如,要训练一个能够实现 AI 对话的模型,人们需要准备\regcolor{巨量}现实生活中的对话,模型从它们中学习到词句间的规律,从而逐渐具有类似人类对话的能力。

\begin{figure}[htb!]
  \centering
  \includegraphics[width=.9\textwidth]{assets/surpass/Model_training.pdf}
  \caption{模型的训练}
  \label{fig:Model_training}
\end{figure}

将训练好的模型加以包装,就能得到可供用户使用的 app 或网站产品,人们可以在 app 或网页上便捷地使用 AI 模型进行推理。2022 年末,美国人工智能技术公司 OpenAI 基于其所研发的「GPT」系列模型,上线了一款 AI 聊天机器人「ChatGPT」,引起全球轰动。随后,谷歌、Meta、微软等行业巨头纷纷推出了自己的 AIGC 模型;与此同时,许多国内人工智能研究机构、高校和企业也紧随其后,各种国产模型纷至沓来。这些产品各自有着不同的特色和应用场景,琳琅满目,让人目不暇接。下图展示的就是 ChatGPT 的界面。

\begin{figure}[htb!]
  \centering
  \includegraphics[width=.9\textwidth]{assets/surpass/ChatGPT.png}
  \caption{ChatGPT 的主页}
  \label{fig:ChatGPT}
\end{figure}

在国内一众 AIGC 模型中,由 AI 企业智谱和清华大学联合开发的 GLM 模型,生成内容的质量和速度都相当不错。智谱推出的「智谱清言」平台,提供了包括 GLM 在内的多种 AIGC 模型供用户使用。下面,我们以智谱清言为例,来一睹 AIGC 的魅力。

\section{体验 AIGC 的魅力}

\subsection{试试 ChatGLM!}

访问智谱清言的官方网站(\url{https://chatglm.cn}),按提示注册账号并登录。登录后,你可以看到平台的主界面,如下图所示。

\begin{figure}[htb!]
  \centering
  \includegraphics[width=.9\textwidth]{assets/surpass/Zhipu_main.png}
  \caption{「智谱清言」的主界面}
  \label{fig:Zhipu_main}
\end{figure}

主界面最左方提供了不同的 AIGC 应用供我们选择,包括基于 GLM 模型的 ChatGLM 对话、AI 画图、数据分析等。默认情况下,我们会进入 ChatGLM 对话应用。界面右侧则是对话区域,我们可以在下方的文本框中向 AI 发送消息,对话的历史则会在上方显示。我们试着向它提出一些简单的问题,通常便可以得到一个详实的回答。

\begin{figure}[htb!]
  \centering
  \includegraphics[width=.8\textwidth]{assets/surpass/Zhipu_q1.png}
  \caption{让 AI 帮你回答问题}
  \label{fig:Zhipu_q1}
\end{figure}

一般来说,现今的对话模型都有一定的「记忆能力」,可以记住我们刚刚聊过什么。因此我们可以向它追问,或者提出一些与之前对话相关的问题,模型会根据之前的对话内容,给出更加详细的回答。当然,我们也可以直接询问它「我之前的问题是什么?」,来验证它的记忆能力。

\begin{figure}[htb!]
  \centering
  \includegraphics[width=.8\textwidth]{assets/surpass/Zhipu_q2.png}
  \caption{追问模型的记忆}
  \label{fig:Zhipu_q2}
\end{figure}

有时,我们不希望模型的回答受到之前对话的影响,这时点击【新建对话】按钮,便可以开始一个全新的对话,不再受之前对话记忆的影响。当然,在界面左方的【历史记录】面板上,我们也可以查看之前的对话记录,甚至继续之前的对话。

\begin{figure}[htb!]
  \centering
  \includegraphics[width=.9\textwidth]{assets/surpass/Zhipu_q3.png}
  \caption{进行多个对话}
  \label{fig:Zhipu_q3}
\end{figure}

你可能已经注意到了,上面的例子中,我们提出的问题都比较偏向于常识,或者说只涉及通用的原理或技术。由于模型是使用海量的数据训练的,因此对于这类问题,模型的回答通常比较准确的——可以理解为,\regcolor{在训练的阶段,模型就已经见识过我们提出的问题或者类似问题了}。但是,如果我们提出一些比较专业、具体,或者提出有关新兴事物的问题,由于训练数据中并没有相关的知识,凭借模型自己的能力是无法回答的。为了解决这一问题,许多 AI 模型在面临此类问题时,会选择上网搜索相关知识,然后给出一个基于搜索结果的回答。

\begin{figure}[htb!]
  \centering
  \includegraphics[width=.8\textwidth]{assets/surpass/Zhipu_q4.png}
  \caption{模型的搜索能力}
  \label{fig:Zhipu_q4}
\end{figure}

在上图中,我们向模型提问「国产芯片最新发展现状」。由于这个问题涉及到了最新的技术发展,模型并没有相关的知识,因此它会选择上网搜索相关的信息。从图中我们可以看到,模型一共阅读了 5 篇互联网的相关报告,将它们总结为答案。这种搜索能力,使得模型在回答问题时,能够给出更加详细、准确的答案。

除了让 AI 帮我们回答各种问题之外,我们还可以让 AI 帮我们撰写我们需要的文章。比如,我们可以让模型帮我们写一封长度在 500 字左右的申请读研的自荐信:

\begin{figure}[htb!]
  \centering
  \includegraphics[width=.8\textwidth]{assets/surpass/Zhipu_q5.png}
  \caption{AI 写的自荐信}
  \label{fig:Zhipu_q5}
\end{figure}

从上图中我们不难发现,这封 AI 撰写的自荐信在结构和格式上大致正确,包括了自我介绍、学术背景、科研经历、个人特长等部分;而在内容上,由于我们并没有提供给模型具体的撰写细节,每个部分的具体内容都是由模型自由发挥而来。这一特点,决定了我们利用 AI 的方式——我们可以将 AI 作为工具,让其帮我们生成基础的构架,再根据自己的需求进行修改和完善。

我们也可以选择在向模型「布置任务」时,提供给它更多的细节,如此一来,模型就会选择性地按我们的要求撰写。以上面的「写自荐信」为例,我们可以额外提供给它一些具体的细节,比如我们的专业、研究方向和个人特长等。这样,模型就会尝试把这些信息融入到生成的文章中,使得文章更加符合我们的需求。

\begin{figure}[htb!]
  \centering
  \includegraphics[width=.8\textwidth]{assets/surpass/Zhipu_q6.png}
  \caption{AI 写的自荐信,但是更多细节}
  \label{fig:Zhipu_q6}
\end{figure}

像 ChatGLM 这样的模型,它们的训练数据中除了对话、文章语料外,还有着许多代码片段,所以我们还可以向模型询问一些关于编程的问题。例如,我们可以让模型帮我们写一个简单的 Python 程序,或者解释一些代码的含义。对于一些编程新手来说,这是一个非常不错的学习方式。

\begin{figure}[htb!]
  \centering
  \includegraphics[width=.75\textwidth]{assets/surpass/Zhipu_q8.png}
  \caption{让模型写快速排序}
  \label{fig:Zhipu_q8}
\end{figure}

除了 ChatGLM 对话模型,智谱清言平台还提供了许多其他的 AIGC 模型供我们使用。例如,我们在网站左侧栏中选择「AI 画图」,然后在右方输入我们希望绘制的内容,还可以指定一些细节,比如画面的主题、色调、风格等,模型会尝试根据这些信息生成一幅符合我们要求的图画。

\begin{figure}[htb!]
  \centering
  \includegraphics[width=.75\textwidth]{assets/surpass/Zhipu_q7.png}
  \caption{AI 画图}
  \label{fig:Zhipu_q7}
\end{figure}

当然,在这个百花齐放的 AI 时代,除了智谱清言平台上的 ChatGLM 模型外,市面上还有诸如 DeepSeek(\url{https://chat.deepseek.com/})、豆包(\url{https://www.doubao.com/})、Kimi(\url{https://kimi.moonshot.cn/})等许多优秀的国产 AI 模型和平台供我们使用,它们的使用方式与上文介绍的智谱清言大同小异,如果你有兴趣,不妨自行上网搜索,一一体验。

\subsection{让 AI 更容易理解我们的意图}

尽管今天各种 AI 模型的能力已经相当强大,但当我们尝试将它们实际应用到工作和学习中时,会发现 AI 经常不理解我们的意图——有时 AI 生成的内容答非所问,有时 AI 忽略了我们的一些细节要求。为了尽量避免这种情况,我们可以尝试优化我们的「提示词」。

「提示词」(prompt)指的是我们向 AI 提出问题或者任务时所使用的文字,前文中诸如「请告诉我国产芯片发展现状」「我是……写一篇自荐信」等都是提示词。人们发现,对于相同的任务,提示词设计的好坏会直接影响到模型的表现。设计提示词是一项复杂的工作,不过,就我们这样的简单应用而言,我们通常只需要注意以下几点:

\begin{itemize}
  \item \regcolor{准确、清晰、通顺}:在日常生活中与他人沟通时,准确、清晰、通顺的语言能够极大地提高我们的沟通效率,在设计与 AI 对话的提示词时亦是如此。比如,「请告诉我国产芯片最新发展现状」就要比「国产芯片怎么样」更加合适,而「请帮我设计一份《计算机系统》教案,共 32 课时,主要集中介绍体系结构知识」往往也会取得比「写一份计算机系统教案」更好的效果。
  \item \regcolor{构造具体场景}:在设计提示词时,我们可以尽量构造一个具体的场景,将我们希望让模型注意到的信息逐项给出。在前文「写自荐信」环节提到的「提供给模型更多的细节」,本质上便是在构造具体的场景,并在其中提供更多背景信息。此外,我们还可以选择对 AI 进行角色引导,比如,如果我们希望让 AI 帮我们总结一些有关金融方面的文章,「你是一位金融方面的专家,请阅读下面的文章并帮我总结它们」便是一个不错的提示词开头。\CJKsout*{或者做一些角色扮演之类的事,比如「你是一只猫娘……」。}
  \item \regcolor{给出样例}:有句话说得好:「举例是理解的试金石。」对于一些比较复杂的任务,例如生成解决具体问题的代码,我们可以在提示词中给出样例,这样模型往往可以快速理解我们的意图。比如,如果我们需要让 AI 帮我们编写程序,将字符串中的数字提取出来并求和,我们可以在提示词中给出一个简单的样例:
    \begin{quoting}
      请帮我写一个 Python 程序,将字符串中的数字提取出来并求和\par
      \phantom{text}\par
      示例输入:s194ab3cd12\par
      示例输出:209\par
      解释:字符串中的数字有 194、3 和 12,它们的和为 209\par
    \end{quoting}
    对于一些复杂的任务,相比于绕来绕去地描述任务要求,给出样例往往更加直接、有效。
\end{itemize}

提示词优化也是一门学问,需要我们在实践中不断摸索。在使用 AI 模型时,我们可以尝试不同的提示词,观察模型的回答,然后根据回答的效果来调整我们的提示词。若你对提示词的设计还有兴趣,不妨上网搜索「提示词工程(prompt engineering)」,了解更多关于提示词设计的技巧。

\section{从加减乘除到 AI 对话}

体验过 AIGC 的魅力之后,你是否好奇,如此神奇的技术究竟是如何实现的?计算机的本质是计算,但从「加减乘除」的计算到人工智能的应用,其间到底经历过怎样的跨越?本节将带着你慢慢揭开 AI 那神秘的面纱,跟随我们的步伐,人工智能背后的基本原理将随之展现。

\subsection{AI 推理的本质——「预测」}

\regcolor{AI 模型推理的本质,一言以蔽之——「预测」。}「预测」指的是\regcolor{依据某种「经验」,根据已知的信息去推测未知的信息}。例如,天气预报是一种预测,人们的「经验」来自于长期对天气的观察和气象学的发展,「已知的信息」则是人们收集到的各种气象数据。借助经验和已知信息,气象学家们可以预测未来的天气情况——这便是「未知的信息」。

以 AI 对话为例,在模型训练过程中,模型通过大量的语料数据,「学习」人类语言的统计规律,这是「经验」。它们让模型明白,「你好」之后通常会伴随另一句问候,「你是谁」之后通常会伴随一句自我介绍;「今天天气真」之后大概率是「好」「不好」等词汇,而「美国的首都是」之后要么是「华盛顿」要么是一个问号。最终,借助这些统计规律,模型能够做到给定任意的一串文字,预测出最可能接在后面的一个词汇\footnote{这个「词汇」的概念更为广泛,通常称为 token,包括字、词、符号等多种语言单元。},完成推理。

\begin{figure}[htb!]
  \centering
  \includegraphics[width=.6\textwidth]{assets/surpass/GPT_is_predicting.pdf}
  \caption{对话模型的预测过程}
  \label{fig:GPT_is_predicting}
\end{figure}

在预测出第一个词汇之后,模型会将这个词汇作为「已知信息」,然后继续预测下一个词汇。这样,模型就可以不断地生成出一段连贯的文本。这种「逐词预测」的方式,便是模型生成文本的基本原理。当然,这个过程中会有诸多的调整和优化,以生成更加符合人类语言习惯的连贯文本。

\begin{note}
  那么,不妨自己想象一下,如果你需要设计一个 AI 绘画模型,它应当要学习什么「经验」?它的「已知信息」又是什么?它又是如何「预测」出一幅图画的呢?
\end{note}

从这样的视角来看,我们就不难明白,AI 模型的关键便是这套预测算法的设计、训练、验证和优化。还是以 AI 对话模型为例:我们如何设计这套预测下一个词的算法?我们用什么样的语句去训练它?如何训练?训练完成之后怎么验证它对话的效果?又如何进行优化?解决了这些问题,一个优秀的模型才能诞生。

\subsection{从价格预测看模型训练}

诸如 AI 对话那样的模型,其内部的结构和算法极为复杂(\url{https://arxiv.org/abs/1706.03762} 论述了大模型的底层细节,闲着没事可以看一看),若用它来介绍 AI 的内部原理,这份教程就和专业教材一般令人头大了。那么,我们不妨选择一个简单得多的预测问题——「价格预测」来研究。在理解了价格预测问题之后,我们就能以小见大,举一反三地理解 AI 训练和推理的过程。

\subsubsection{线性回归模型}

价格预测问题是:假设有一种金属首饰,其价格主要由重量决定,我们现在想从首饰有多重来预测它应该会卖多少钱。这个问题里,重量就是「已知的信息」,售价则对应「未知的信息」为了获取「经验」来解决这个预测问题,我们购买了市面上 20 款不同的首饰,它们的重量与价格如下表所示:

\begin{table}[htb!]
  \centering
  \caption{某首饰的市场行情}
  \label{tab:prices-of-jewelry}
  \begin{tblr}{
    colspec = cccccc,
    row{1} = {valign=m, fg = white, bg = missing, font = \bfseries},
    row{even} = {MissingSkyBlue},
    vline{4} = {solid},
  }
    \toprule
    序号 & 重量(克) & 价格(元) & 序号 & 重量(克) & 价格(元) \\
    \midrule
    1    & 68.42      & 4098       & 11   & 85.42      & 4698       \\
    2    & 80.06      & 4598       & 12   & 67.02      & 3948       \\
    3    & 72.19      & 4098       & 13   & 69.76      & 3948       \\
    4    & 68.14      & 3998       & 14   & 94.79      & 5248       \\
    5    & 59.66      & 3348       & 15   & 34.97      & 2048       \\
    6    & 75.21      & 4298       & 16   & 36.10      & 2348       \\
    7    & 60.63      & 3398       & 17   & 31.42      & 2098       \\
    8    & 92.42      & 5298       & 18   & 88.28      & 4948       \\
    9    & 97.46      & 5398       & 19   & 84.47      & 4898       \\
    10   & 56.84      & 3298       & 20   & 90.90      & 5098       \\
    \bottomrule
  \end{tblr}
\end{table}

这些数据会用来「训练」我们的模型。我们可以将这些数据绘制成散点图,横轴是首饰的重量(克),纵轴是售价(元)。如下图所示:

\begin{figure}[htb!]
  \centering
  \includegraphics[width=.65\textwidth]{assets/surpass/Prices_of_jwewlry.pdf}
  \caption{首饰重量与价格的关系}
  \label{fig:Prices_of_jwewlry}
\end{figure}

显然,重量和价格之间的关系可以用一条直线来刻画——具体地说,如果把重量记作 $x$,把对应的售价记作 $y$,那么它们之间的关系就可以用方程
\[
y = ax + b
\]
来描述。\regcolor{这个方程就是我们的「模型」,其中 $a$ 和 $b$ 是这个模型的「参数」。}$a$ 和 $b$ 的值决定了这条直线长什么样,也就决定了我们的模型对价格的预测效果。

现在,我们的训练目标就是:对上面那 20 个数据\regcolor{进行「学习」,来找到最合适的参数},这样我们的模型就可以胜任价格预测任务了。这就是「线性回归」模型的一个例子,它是一种极为常用的数学模型。

\begin{note}
  你可能会认为(或者听人说)「线性回归之所以叫线性回归,是因为它的图像是一条直线」,但其实不然,线性回归的「线性」,在于关于自变量的各个函数是线性组合的,或者说,各自乘一个参数再加起来。也就是说,$y(x) = ax^3 + b \ln x + c \mathrm{e}^x$ 也是线性回归,而 $y(x) = x^a + b$ 则不是。
\end{note}

\subsubsection{「损失」函数}

在这个模型中,合适的参数,能够使得某个重量下的\regcolor{模型预测的价格和实际价格之间尽可能接近}。那么,我们先把那 20 款首饰的重量 $x_i$($i=1,2,\cdots,20$)逐一代入我们的模型,预测出对应的价格
\[
\hat{y_i} = ax_i + b,\qquad i=1,2,\cdots,20\text{。}
\]
显然,我们会希望所有的 $\hat{y_i}$ 和相应的实际售价 $y_i$ 之间的距离 $|\hat{y_i} - y_i|$ 都能越小越好。怀着这样的想法,我们可以把所有的距离的平方加起来(避免麻烦的绝对值),得到差距的平方和,称之为「损失」$L$,即
\[
L = \sum_{i=1}^{20} (\hat{y_i} - y_i)^2 = \sum_{i=1}^{20} (ax_i + b - y_i)^2
\]
至此,我们的任务就变成了:找到一对 $a$ 和 $b$ 的值,使上面的损失 $L$ 最小。

观察上面的式子,在那 20 个样本点给定,也就是说 $x_i$ 和 $y_i$($i = 1, 2, \cdots, 20$)都是定值的情况下,$L$ 只和 $a$ 与 $b$ 有关——$L$ 是一个关于 $a$ 和 $b$ 的函数,我们记作 $L(a, b)$,称为「\regcolor{损失函数}」。

那么,\regcolor{训练模型,就是求损失函数 $L(a, b)$ 的最小值点}。

\subsubsection{梯度下降}

如何找到这个函数的最小值点呢?在这个例子中,损失函数
\[
L(a, b) = \sum_{i=1}^{20} (ax_i + b - y_i)^2
\]
比较简单,所以我们可以直接通过求偏导数\footnote{对多元函数而言,我们一次只对一个变量求导数,谓之「偏导数」。}零点的方式找到最小值点。但是,当模型的参数量不是 2 而是 2000 甚至 2 千万时,当模型不再是线性回归而是由更复杂的函数构成时,直接解出最小值就不太现实了。我们需要寻找一个通用的方法,对于再复杂的损失函数,都可以有效地找到我们需要的最小值点——哪怕不十分精确,但是足够接近就可以。这个方法就是「梯度下降」。

\begin{note}
  对于线性回归问题,直接求出最小损失值点的方法叫做「最小二乘法」。
\end{note}

现在把刚刚的损失函数放到一边,考虑一个看起来坑坑洼洼的二元函数 $f(x,y)$,图中左侧是它的三维图像,右侧是图像从上往下看的样子,我们的目标是找它的最小值点。先随意挑一个点 $\mathbf{P}_0 = (x_0, y_0)$ 作为起点:

\begin{figure}[htb!]
  \centering
  \includegraphics[width=.7\textwidth]{assets/surpass/GD_1.png}
  \caption{挑选一个下降的起点}
  \label{fig:GD_1}
\end{figure}

显然,这个随意挑选的点并非该函数的最小值点。但是,我们可以借助它来寻找最小值点可能的方向。我们计算 $f(x,y)$ 在 $\mathbf{P}_0$ 处的梯度方向\footnote{倒三角算符的定义是\[\nabla f(x,y) = \left( \frac{\partial f}{\partial x} , \frac{\partial f}{\partial y} \right),\]所以它就只是求了两下偏导数,并将之作为向量的两个分量而已。} $\nabla f(x_0, y_0)$——梯度方向,类似于一元函数的导数,指的是函数值在某一点上升最快的方向。既然我们需要让函数值变小,那此时\regcolor{背朝梯度方向走一步},不就正好向着下降最快的方向移动了吗?于是我们令 $\mathbf{P}_1 = \mathbf{P}_0 - \alpha \cdot \nabla f(x_0, y_0)$,其中 $\alpha > 0$:

\begin{figure}[htb!]
  \centering
  \includegraphics[width=.7\textwidth]{assets/surpass/GD_2.png}
  \caption{沿着梯度的反方向走一步}
  \label{fig:GD_2}
\end{figure}

不难发现,我们得到了一个比 $\mathbf{P}_0$ 更靠近函数最小值的 $\mathbf{P}_1$。上面「走一步」式子中的 $\alpha$ 是一个「步长」参数,它影响着我们每一步「走」的距离。不断这样走下去,我们就可以逐渐接近函数的最小值点:

\begin{figure}[htb!]
  \centering
  \includegraphics[width=.7\textwidth]{assets/surpass/GD_3.png}
  \caption{梯度下降的过程}
  \label{fig:GD_3}
\end{figure}

事实上,如果函数满足某些特性,在某一局部存在最小值点,通过选择好调整合适的步长参数 $\alpha$,不断地「走」,就可以近似找到这个最小值点。这个过程即是「梯度下降」,步长参数 $\alpha$ 称为「学习率」,每走一步称为「训练一步」或「迭代一步」,而整个寻找最小值的过程,就是 AI 模型开发过程中最为耗时的过程之一——训练。

打个不恰当的比方,梯度下降的过程就像在一片山川中随机选一个点,放一个球,看着它在重力的作用下滚到山谷里的过程。回到价格预测问题的那个损失函数 $L(a, b)$,即使它非常复杂,只要它足够光滑并满足一些条件,我们就可以从一个随机的起点 $(a_0, b_0)$ 开始,用梯度下降的方法逐步找到它的最小值点。模型训练过程的核心就是这样梯度下降的过程。

\begin{note}
  当然,实际的训练过程中还会对许多细节进行优化。譬如,上文中提到的学习率 $\alpha$ 的取值,对训练的效率有着至关重要的影响。选择合理的 $\alpha$,或动态地调整 $\alpha$,就能有效地提升训练效果。此外,当样本点的数量不是 20 个而是 2000 甚至 200 万个时,对样本点的处理方式亦十分关键。又或者,一个损失函数有许多局部的最小值点,这些值有大有小,想要找到全局的最小值,可能还需要在不同的地方取初值,多次尝试……不过,万变不离其宗,这些优化的方法都是围绕着「梯度下降」这个核心展开的。
\end{note}

在一切顺利的情况下,\regcolor{训练一定步数后,模型损失会在最小值附近徘徊}。此时,我们可以「见好就收」,停止训练并固定所有参数。记这时模型的参数为 $\hat{a}$ 和 $\hat{b}$,我们就得到了训练好的模型
\[
y = \hat{a}x + \hat{b}\text{。}
\]

现在,将首饰的重量代入 $x$,模型就能「推理」出一个预测的价格 $y$。这个预测值会较好地符合学习到的规律。

\subsection{神经网络和深度学习}

神经网络是仿照人的神经系统所设计的一种模型结构。人类的神经系统由无数的神经元组成,每个神经元可以接收、处理来自其他多个神经元的输入信号,随后传递出去。神经元之间相互连接,织成一张巨网,令它能够高效地进行极为复杂和多样的信号处理,最终成就了我们的意识。

\begin{figure}[htb!]
  \centering
  \includegraphics[width=.5\textwidth]{assets/surpass/NeuroNetwork.pdf}
  \caption{简单的神经网络}
  \label{fig:NeuroNetwork}
\end{figure}

仿照这样的结构,我们可以设计出「人工神经元」,它只进行一些简单的运算,接收多个输入,产生一个输出。当大量不同种类的神经元相互逐层连接时,就能形成一个非常巨大的「神经网络」,如\autoref{fig:NeuroNetwork} 所示。这其中,神经元本身和那些连接线都是可以调整的「参数」。尽管每个神经元都只进行简单计算,但大量的神经元构成的网络,有着巨大的参数量,因此有着非常强的拟合能力。这样的神经网络模型广泛应用于人工智能领域。由于现在应用的神经网络层数较多,「深度」较深,因此基于神经网络的相关技术又被称为「深度学习」。

无论是上文介绍的最简单的线性回归模型,还是复杂的神经网络,再到各种各样的 AIGC 模型,对它们的训练,本质上并没有区别。不管模型多么复杂,参数数量多大,训练模型仍然需要构造合适的「损失」函数,收集大量的训练数据,通过梯度下降或一些更好的方法,逐步调整模型参数,让损失函数的值逐渐减小。这一切未曾改变,只是在更加复杂的模型和更加庞大的数据集上进行罢了。

读完上文的介绍,你对 AI 模型从诞生到应用的过程应该有了一个大致的了解。下面我们总结一下这个过程:

\begin{itemize}
  \item \regcolor{需求分析与模型设计}:我们要根据实际的需求设计出合理的 AI 模型,确定模型的结构等技术细节。在上面价格预测的例子中,我们通过观察数据特征,选择了线性回归模型;而面对更加复杂的场景,更加复杂、能力更强的模型也会被设计出来。
  \item \regcolor{数据采集与预处理}:我们需要收集大量用于训练的数据。对于价格预测,那 20 个样本就是我们全部的训练数据。现实中,我们不仅需要收集海量数据,还要对数据进行清洗、去噪、标注等处理。
  \item \regcolor{模型训练}:将训练数据送入模型,梯度下降调整模型参数,再重复这个过程——就这。这一环节是整个 AI 模型开发中最为耗时的一个环节,但是它的本质,与我们上面讲的线性回归模型训练过程并没有太大的区别。
  \item \regcolor{模型验证}:在训练结束后,我们还需要想办法验证我们模型的效果。常用的方法是再另外收集一些数据,将数据送入模型进行预测并与实际值进行比较。如果模型的预测效果不好,我们还需要回到「模型训练」环节,调整模型的结构、参数等。
\end{itemize}

这个过程中,每一步都需要大量的实验和探索,背后则是无数科研人员的努力。

\section{大模型、GPU 和 AI 芯片}

现在,我们已经对 AI 内在的原理有了初步的认识——AI 模型相当于一套极为复杂的数学运算过程,可以根据它拿到的信息进行预测。在继续探索后面的内容之前,让我们暂时来思考一个问题:为什么像智谱清言、DeepSeek 或是 ChatGPT 这样的 AI 平台,都需要联网在线运行呢?明明我们自己的电脑、手机都能运算,为什么这些 AI 仍然只能供我们在浏览器中在线体验呢?

\begin{note}
  你可能在手机或电脑上下载过这些 AI 平台对应的 app,但是它们的计算都是运行在远处的服务器上的。如果你将手机或电脑的网络断开,它们就无法运行了。
\end{note}

\begin{figure}[htb!]
  \centering
  \includegraphics[width=.6\textwidth]{assets/surpass/AI_running_on_server.pdf}
  \caption{在服务器上运行的 AI}
  \label{fig:AI_running_on_server}
\end{figure}

带着这个问题,让我们一起来看看今天,大模型的「烦恼」。

\subsection{大模型的烦恼}

人们已经发现,不断增大 AI 模型的规模,增加参数的数量,就能显著地提升模型的效果。例如前文我们体验的 GLM 模型,它的参数量可达上千亿,其开源版本 GLM-4-9B 的参数量也能达到 90 亿。其他一些大模型\footnote{这里就指规模很大的模型,不过现在「大模型」一词常被用来特指「大语言模型」(Large Language Model,简称 LLM),即我们文中所谓的「AI 对话模型」。}的参数量也能达到上百亿甚至更多。下表展示了目前一些流行大语言模型的参数量。

\begin{table}[htb!]
  \centering
  \caption{一些大模型的参数量}
  \label{tab:param-of-llms}
  \begin{tblr}{
    colspec = ccc,
    row{1} = {valign=m, fg = white, bg = missing, font = \bfseries},
    row{even} = {MissingSkyBlue},
  }
    \toprule
    模型名称 & 参数量 & 开发机构 \\
    \midrule
    GPT-3        & 1750 亿               & OpenAI         \\
    GPT-4        & 未知,据称达 1.8 万亿 & OpenAI         \\
    GLM-3        & 60 亿至上千亿         & 智谱和清华大学 \\
    GLM-4        & 90 亿至上千亿         & 智谱和清华大学 \\
    通义千问 1.5 & 5 亿至 1100 亿        & 阿里云         \\
    通义千问 2.5 & 5 亿至 720 亿         & 阿里云         \\
    Llama 2      & 70 亿至 650 亿        & Meta           \\
    Llama 3      & 80 亿至 4 千亿        & Meta           \\
    DeepSeek V3  & 6710 亿               & 深度求索       \\
    DeepSeek R1  & 15 亿至 6710 亿       & 深度求索       \\
    \bottomrule
  \end{tblr}
\end{table}

在上一节中我们已经知道,对模型进行训练,本质是对它的参数进行调整优化。然而,对于这些参数量轻松破亿的大模型来说,训练它们是一件非常困难的事情——巨大的参数量,意味着巨大的计算量。试想,按上一节的思路,如果我们逐个对这些参数求导来进行梯度下降,就算 CPU 速度再快,也不知道得算到何年何月才能完成一轮训练。另外,大模型的推理同样也是一个挑战。寻找高效的运算方法,就成为了一个重要的突破口。

\subsection{GPU 的「副业」}

我们仔细分析模型中训练和推理运算的过程,可以发现它的两个特征:

\begin{itemize}
  \item \regcolor{可并行}:在这些过程中,模型需要处理大量的数据。对这些数据之间的计算往往相互独立,因此我们可以同时对多个数据进行计算来提高效率。此外,许多模型在设计时也引入了可分离的结构。这些特点使得\regcolor{模型运算的过程可以很好地并行化,将一个任务拆分成多个小任务同时进行}。
    \begin{note}
      神经网络正是可并行的,想想为什么?
    \end{note}
  \item \regcolor{以矩阵运算为主}:在具体的模型实现中,诸如求解梯度、梯度下降等操作,通常都化为了矩阵运算,包括矩阵的乘法、求逆、分解等。另外,模型中大量的数据和参数亦以矩阵的方式存储。
\end{itemize}

这两个特点使得\regcolor{模型运算的过程无法在 CPU 高效进行}——我们的 CPU 通常只有几个核心,能够同时执行的计算任务有限;而且,CPU 是一种通用计算芯片,它主打「会得多」,而不是「算得快」,针对矩阵运算这样特定的计算任务,CPU 并没有进行过多的优化。自然而然,我们需要寻找一个更加适合 AI 模型的计算设备,而用来处理图形的显卡(GPU)就进入了我们的视野。

GPU 作为一种专门用来进行图形处理的设备,在设计上有着与 CPU 不同的特点。首先,为了满足图形处理需要的极高同步率,\regcolor{GPU 拥有大量的计算单元,它们可以同时执行大量计算任务},这样我们在用 GPU 打游戏、看视频时,才不会出现卡顿和画面撕裂。其次,由于图形最终以像素点组成的图像在屏幕上展现,而图像的本质是矩阵,因此 \regcolor{GPU 在设计上就针对各种矩阵运算做了大量的优化}。GPU 的这两个特性给我们带来了更好的视觉体验,也让它成为了模型运算的理想设备——让 GPU 干点儿跑模型的「副业」,可能会有意想不到的效果。

2007 年,GPU 巨头英伟达推出了 CUDA 平台,让人们可以利用这一平台在英伟达的 GPU 上进行各种计算任务。随后的 2009 年,由 AMD 和苹果等联合推出的 OpenCL 平台则在更多品牌的 GPU 上带来了类似的功能。与这些平台相对应的各种机器学习框架也应运而生,极大地方便了 AI 模型的设计、训练和应用的过程。时过境迁,CUDA 技术伴随着英伟达的迅速发展不断占据着市场,最终,英伟达 GPU 成为了 AI 模型训练和推理的标配。

\subsection{AI 芯片的崛起}

虽然 GPU 在 AI 模型运算中具有较好的性能,但它的本职工作仍然是图形处理,跑 AI 模型终究是一种副业。于是,人们开始在 GPU 的基础之上,设计一种专门用来进行 AI 运算的芯片,这就是「AI 芯片」。与 GPU 相比,AI 芯片不再带着图形处理的包袱,也就无需考虑与图形处理相关的部分,而可以把更多的资源投入到优化 AI 模型的训练和推理上。从 GPU 到 AI 芯片,副业转正。

作为老牌 GPU 厂商的英伟达,早在 2007 年就推出了专门用来进行 AI 运算的 GPU 系列——Tesla。该系列产品专门针对 AI 运算进行了优化。此后,英伟达在这一产品线不断推陈出新。目前较新的 V100、A100、H100 等产品,有着非常强大的性能,成为了 AI 运算的标杆。这些芯片虽然也时常被称为「GPU」,一些操作系统也把它们作为 GPU 处理,但是它们已经没有了视频输出等图形处理的功能。类比于「显卡」一词,大家把这样的产品称为「计算卡」。下图展示的便是 V100 计算卡。

\begin{figure}[htb!]
  \centering
  \includegraphics[width=.7\textwidth]{assets/surpass/V100.jpg}
  \caption{一张英伟达 V100}
  \label{fig:V100}
\end{figure}

与此同时,其他厂商,尤其是我国的科技公司亦不甘落后。谷歌、华为、阿里等公司都推出了自己的 AI 芯片,如谷歌的 TPU、华为的昇腾系列、阿里的 Ali-NPU 等。它们的出现不仅为 AI 计算提供了更多的选择,为市场注入了更多的活力,同时也推动着 AI 技术整体的发展。

\begin{note}
  可以想象,在 AI 越来越重要的时代,GPU 和计算卡这种「算力」设备的重要性与日俱增——事实上,它们已经成为了大国之间政治博弈的筹码。2023 年 10 月 17 日,美国商务部工业和安全局(BIS)公布新的先进计算芯片、半导体制造设备出口管制规则,禁止向我国出口具有较强计算能力的计算设备,包括顶级游戏显卡 RTX 4090、上文提到的计算卡 A100 等。这更加凸显了在 AI 芯片领域实现自主可控的重要性。
\end{note}

由于现在多数 AI 模型都由神经网络驱动,人们常常把这种 AI 芯片称为「神经处理器」(Neural Processing Unit,简称 NPU),以和 CPU、GPU 相对应。

\subsection{本地部署成为新宠儿}

现在,让我们回到本节开头的那个问题——为什么今天大多数的 AI 都运行在远方的服务器上,而不是我们自己的设备上呢?答案显而易见——我们自己的手中的算力,通常不足以运行 AI 模型。

以智谱清言所使用的 GLM 模型的开源版本 GLM-4-9B 为例,其拥有 90 亿参数。如果我们想使用 GPU 来运行它,需要近 20 GB 的显存(以 BF16 加载),而今天拥有这么多显存的显卡均价格不菲。又比如当下大火的 DeepSeek 模型,其「满血版」拥有 6710 亿参数,需要同时使用多张显存巨大的专业计算卡(价格非常昂贵)才能流畅运行,这显然不是一般人所能承受的。

为了减少显存的占用,人们提出了「低比特量化」的思路,通过降低参数的精度,来减小存储它们所需要的体积。然而,即使使用了量化技术,AI 模型对显存的需求依然不可小觑。此外,即使不考虑显存的需求,AI 对显卡本身的性能也提出了相当大的考验——毕竟,一个每秒只能生成半个字的 AI 助手,观感上就不是很好,更别说实用了。这些因素让 AI 的「本地部署」困难重重。

不过,尽管大型模型的本地部署相当困难,但是「将 AI 模型本地部署」这事本身有着许多的好处:首先,本地部署解决了「随时使用」的问题。当网络连接不通畅或同时使用的人数过多时,那些在线的 AI 平台就会「服务器繁忙,请稍后再试」了。如果将 AI 模型运行在自己的设备上,就完全不必担心这样的情况。同时,访问 AI 的延迟也能降低。此外,本地部署还可以有效防止数据外泄——假如你是某企业的负责人,如果你的员工每天都在借助网上的 AI 平台来解决工作问题,你是否会担心员工不小心将企业的敏感信息发给它们?通过将 AI 模型本地部署,就能实现数据的可控。

如今,越来越多的研究开始聚焦于小尺寸 AI 模型——它们的参数量更小,对硬件资源的需求也更低。与此同时,手机与电脑处理器正逐步集成 NPU,以提升 AI 计算的效率。目前,众多手机和电脑厂商正采用「小模型本地部署 + 大模型在线服务」的策略,打造更加智能和通用的 AI 助手。此外,量化技术的进步,使得即便是低端 GPU,甚至纯 CPU,也有能力运行稍大的 AI 模型。如果你对此感兴趣,可以在互联网上搜索\clearglue{}「<模型名称> 本地部署」\restoreglue{},获取更多相关信息。

\begin{note}
  虽然小尺寸模型便于本地部署,但是它们的能力通常就不如大尺寸模型了,这就是所谓的「鱼和熊掌不可兼得」。
\end{note}

\begin{dangerbox}
  目前,几乎所有 AI 模型本地部署的框架都是开源软件,而能够本地部署的都是开源模型——这意味着你不需付出任何软件成本就能使用它们。(当然,设备还是得自己买的。)\regcolor{所有诱导你「付费体验本地部署模型」的(无论是 DeepSeek、Llama、GLM 还是别的什么),都是诈骗。}
\end{dangerbox}

\section{展望人与 AI 的未来}

谈到「人工智能」四个字,你会感到什么?是未来已至,抑或是脊背发凉?无论如何,时代的车轮滚滚向前,我们不必,更无法拒绝 AI 走入我们的生活;展望一个与 AI 共处的未来,则是当下我们需要思考的问题。

\subsection{是敌还是友?}

对于我们来说,AI 到底是我们的助手、我们的朋友,还是我们的敌人、我们的末日呢?

相信通过阅读这一章,你会逐渐开始对 AI 袪魅。就如我们一直强调的那样,看似再神奇的 AI 技术,本质都只不过是基于从海量学习中得到的统计规律进行的某种「预测」而已。一个老生常谈的话题——人类距离真正的人工智能还有多远?这个话题下的「人工智能」,指的是「通用人工智能」(general artificial intelligence,简称 GAI),在预言中,它就像我们人类一样,会思考、有情感、能学习,可以胜任各种精巧的任务……然而,无论是今天已经遍地开花的传统 AI 技术——譬如人脸识别、图像处理——还是那些生成式大模型,它们的本质都是一种「专用人工智能」(narrow artificial intelligence,简称 NAI),也就是说,它们只能干一种活。或许 AIGC 的出现,让这些 NAI 看起来也具有了人类的情感与思维,但其「预测」的本质仍未改变,仍然只是在海量数据支撑下的一种拟合;而当下人工智能技术的发展,本质亦只是改进这种拟合的方式和引入更多的数据罢了。

% 此段与《密码网安》的「完整才是硬道理」一节联动
显然,「意识」作为我们人类独有的特质,是这样的「预测机」难以望其项背的。我们能够在思考的基础上创新,而 AI 只能将文字排列组合,生成一些看似合理的内容;我们具有复杂的语言和情感,但 AI 从未将它们真正理解,遑论给出心心相惜的回馈。更实际一点来说,我们能说着话的同时也顺便唱两句歌,但 AI 还要在各个模型之间切来切去。即使是今天最先进的大语言模型,错误的回答与偏颇的判断仍屡见不鲜;而人类无穷无尽的想象力、领导力和逻辑能力,是人工智能技术永远也无法完全实现的。

虽然 AI 无法真正拥有意识,不会成为人类的「敌人」,但可以预见的是,AI 的发展会让许多工作职位消失。正如两百年前,蒸汽机带起了工业革命的滚滚烟尘,纺织机的出现变革了人类劳动力的结构,AI 应用的快速普及同样对社会结构、经济模式乃至人们的文化生活都会有显著影响。今天的我们处于这席卷全球的技术洪流之中,更要顺应时代,革新自我——学会将各种各样的 AI 技术用在我们的生活中。让 AI 成为我们的助手,已经成为了我们智能时代之下的一门必修课。

\subsection{争议与危险,掩于智能的面纱}

人类天生就会倾向于信任他们认为比自身厉害的人,并就自己不懂的问题向这些「高人」请教。幼时我们信任我们的家长与老师,从他们那里我们能得到关于未知的答案;长大后我们信任书本与比我们优秀的人们,书本带给我们广阔的知识,优秀的朋友带给我们崭新的视角……但当一个知无不言又似乎学富五车的「谈话对象」出现在我们眼前时,我们自然地就会倾向于「万事先问他」,这就是当下人们面对 AI 对话模型时的感受。但我们一直在强调——\regcolor{AI 大模型的本质是「预测」},这也意味着,\regcolor{AI 很可能犯错}。

虽然 AI 会犯错,但有些错误不是我们能够及时发现的——尤其是在询问我们知识领域外的内容时。\autoref{fig:SolveMaths} 展示了询问某一 AI「请找出所有满足『其二次剩余为 3』的素数 $p$」时它返回的结果,这个过程中,我们质疑了它的回答两次,而它总计回答了 3 个不一样的结果。你可能在想:「这里面只要有对的不就行了吗?」但问题显而易见:你会做这题吗?所以,在询问我们知识领域外的内容时,由于无法判断结果正误,拿到 AI 错误结果的风险大幅上升。

\begin{figure}[htb!]
  \centering
  \includegraphics[width=.95\textwidth]{assets/surpass/SolveMaths.png}
  \caption{让 AI 解数学题}
  \label{fig:SolveMaths}
\end{figure}

这个例子中,AI 给出的三个结果都不一样,而且更搞笑的是,其中第二个是正确答案,第三次生成反而把正确的改成错误的了。(顺便说一下,答案是满足 $p = 12k \pm 1$ 的所有素数,其中 $k$ 是正整数。)可见,一般的数学问题 AI 都无法完美搞定,更不用说让它做「为我们设计一个发动机,给出一套能投产的图纸」这样的工作了。所以,\regcolor{AI 生成的内容,最终还需要人类来验证是否正确、是否有效}。

但是,反过来说,有些别有用心的人更希望 AI 犯错,或者让它生成一些社会公序良俗所不容的东西。曾在社会上引起轩然大波的 Deepfake,虽然不算生成式 AI,但也是一种 AI 加持的「换脸」技术,普罗大众一般会使用它来娱乐,譬如把自己的脸换到影视片段中等等。但凡事均有两面性,不法分子利用它来生成一些名人的影像,影像之中的名人做着他们从来没有,更不可能会去做的事。更有甚者,用它来伪造视频,从而散布谣言、敲诈勒索,进行各种犯罪。

又如一些图像生成 AI,例如 Midjourney、DALL-E 等,可以根据用户的提示词去生成各种类似画作,甚至照片的图像。一方面,这些图像生成 AI 让普通人体验了一把「快速画画」的感觉,用来生成一些有趣的图像,为生活增添不少乐子;另一方面,总有好事之徒想让它们生成一些有争议的图像,例如什么「特朗普被一群警察围捕」。值得一提的是,2024 年 2 月,发表在国际学术杂志《细胞与发育生物学前沿》(\textit{Frontiers in Cell and Developmental Biology})上的一篇论文中,作者通篇使用了 Midjourney 生成的图像作为插图\CJKsout*{(他们甚至懒得把 AI 生成出来的「古神文字」改成正常的英语)},下图就是他们论文中的插图之一。不过,这篇文章已经撤稿,看来学术不端行为又有了新的类别。

\begin{figure}[htb!]
  \centering
  \includegraphics[width=.75\textwidth]{assets/surpass/FigureByMidjourney.jpg}
  \caption{论文中 AI 生成的插图}
  \label{fig:FigureByMidjourney}
\end{figure}

而谈到 AI 生成图像,我们不难想到:既然 AI 的本质是「预测」,那这些 AI 所用模型的训练数据又从何而来呢?这个问题会将我们导向一个有些可怕的想法:这些 AI 的训练者们随意地从网上获取图片用于训练模型,无所谓「版权」二字。从道德上来说,自己的作品未经自己允许被人拿去训练 AI,然后用来仿冒自己的风格,任谁都感到愤怒;从法律上而言,这种无视著作权的行为正是当今世界各国的著作权法都不容的行为。如今,许多创作者都明确声明「禁止将我的作品用作 AI 训练」,可这种声明还是防君子不防小人,AI 行业的伦理规范,仍然需要自我约束、自我鞭策。

\subsection{未来,与 AI 同在}

虽然 AI 技术奇点距离我们似乎还非常遥远,但我们已经可以逐渐在生活的方方面面切实感受到 AI 带来的变化与革新。从简单的加减乘除,到庞大的神经元矩阵,再到百花齐放的 AIGC 模型——已经不再神秘的 AI,是否在你的眼中有了不同的形象?在 AI 的强力辅助下,我们能够减少许多重复而枯燥的工作,也能在与它对谈中激发一些新的灵感。但我们仍然要记住,现在的 AIGC 并不完全可靠,AI 生成的内容,一定要经过人的检验,才能够进一步为人所用。

与 AI 共处,是生在这个时代的人们迫在眉睫的目标,更是我们需要调整自身,与时俱进的动力。适时思考 AI 与人类的关系,虽然不会立马让生活变得更快乐,但说不定能让你在辩论赛中崭露头角,以及为身处将来 AI 遍及大地的社会中增添一些信心。愿 AI 真正成为人类的助手与朋友。

\practice

\begin{enumerate}
  \item 选择一款在线 AI 平台,体验「AI 对话」「AI 绘画」等功能。你还可以对比不同提示词在完成同一任务时的效果差异,以及横向对比多款不同 AI 模型的效果。
  \item 用自己的话说说什么是「训练」和「推理」,以及简要分析为什么模型的训练是相当耗时的过程。
  \item 利用搜索引擎和电商平台,了解目前 GPU、计算卡的市场概况。
  \item 你觉得使用有版权的内容训练 AI 是合乎道德的吗?说说你的见解。
\end{enumerate}