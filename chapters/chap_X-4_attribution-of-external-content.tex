\chapter{借物表}
\label{cha:attribution-of-external-content}

在《你缺失的那门计算机课》中,为便于讲解,我们使用了一些来自互联网的图片等素材。在此,我们按章节逐一列出它们的来源、作者和版权信息,供读者查阅。我们对这些内容的作者表示衷心的感谢。

本书中提及和展示的商标、标志及相关知识产权,无论是否标注有 ® 或 ™ 及其他标识,均归它们各自的权利持有人所有。引用这些商标或标志仅用于识别和介绍相关产品、服务或公司,并无意侵犯其商标或版权。鉴于许多商标和标志图像的版权与授权方式难以界定,我们将它们一并在本页列出,并注明它们的原始来源。

若您对本作品的使用范围有任何疑问,或认为我们对这些内容的使用存在侵权行为,请联系 \href{mailto:missing@criwits.top}{\texttt{missing@criwits.top}}。

\section{各许可协议的链接}

\begin{enumerate}
  \item CC BY 4.0(\url{https://creativecommons.org/licenses/by/4.0/});
  \item CC BY-SA 4.0(\url{https://creativecommons.org/licenses/by-sa/4.0/});
  \item CC BY-NC-SA 4.0(\url{https://creativecommons.org/licenses/by-nc-sa/4.0/});
  \item CC BY 3.0(\url{https://creativecommons.org/licenses/by/3.0/});
  \item CC BY-SA 3.0(\url{https://creativecommons.org/licenses/by-sa/3.0/});
  \item CC BY 2.0 Generic(\url{https://creativecommons.org/licenses/by/2.0/})。
\end{enumerate}

\section{\nameref{cha:computer-and-its-components}}

\begin{enumerate}
  \item 英特尔 i5-13600K CPU 的正面照片(\autoref{fig:Two_CPUs} 左侧)\\
    原图为 \textit{Intel Core i5 13600K}(\url{https://commons.wikimedia.org/wiki/File:Intel_Core_i5_13600K.jpg}),作者是 4300streetcar。\\
    按 CC BY 4.0 协议许可。\\
    我们对原图进行了裁剪。
  \item 英特尔酷睿品牌标识(\autoref{fig:Intel_sticker} 左侧)\\
    相关标识下载自网页「Intel Announces Major Brand Update Ahead of Upcoming Meteor Lake Launch」(\url{https://www.intel.com/content/www/us/en/newsroom/news/intel-announces-major-brand-update-upcoming-meteor-lake-launch.html})。\\
    © 2024 Intel Corporation. Intel, the Intel logo and other Intel marks are trademarks of Intel Corporation or its subsidiaries.\\
    我们对原图进行了裁剪。
  \item AMD 锐龙品牌标识(\autoref{fig:AMD_sticker} 左侧)\\
    相关标识下载自网页「Streaming Powered by AMD Ryzen™ Processors」(\url{https://www.amd.com/en/products/processors/laptop/ryzen/streaming.html})。\\
    © 2024 Advanced Micro Devices, Inc. AMD and AMD Arrow logo, AMD Ryzen™, Ryzen \& Ryzen Logo are trademarks of Advanced Micro Devices, Inc.\\
    我们对原图进行了裁剪。 
  \item 台式机内存条(\autoref{fig:RAMs} 左侧)\\
    原图为 \textit{Samsung DDR4-RAM 20210612 001}(\url{https://commons.wikimedia.org/wiki/File:Samsung_DDR4-RAM_20210612_001.png}),作者是 PantheraLeo1359531。\\
    按 CC BY 4.0 协议许可。
  \item 笔记本电脑内存条(\autoref{fig:RAMs} 右侧)\\
    原图为 \textit{DDR3 SD-RAM SO-DIMM}(\url{https://www.flickr.com/photos/25548012@N02/10852366514/}),作者是 Felix5413。\\
    按 CC BY 2.0 Generic 协议许可。\\
    我们调整了原图的背景色。
  \item NVMe 固态硬盘(\autoref{fig:HDD_and_SSD} 右侧)\\
    原图为 \textit{Samsung 980 PRO PCIe 4.0 NVMe SSD 1TB-top PNr°0915}(\url{https://commons.wikimedia.org/wiki/File:Samsung_980_PRO_PCIe_4.0_NVMe_SSD_1TB-top_PNr%C2%B00915.jpg}),作者是 D-Kuru/Wikimedia Commons。\\
    按 CC BY-SA 4.0 协议许可。
  \item NVIDIA 官网展示的 MSI GeForce RTX 4070 Super 显卡(\autoref{fig:4070_storepage})\\
    截图自网页「MSI GeForce RTX 4070 SUPER 12G GAMING SLIM WUKONG EDITION」(\url{https://marketplace.nvidia.com/en-us/consumer/graphics-cards/msi-geforce-rtx-4070-super-12g-gaming-slim-wukong-edition/})。\\
    © 2024 Micro-Star Int'l Co.Ltd. MSI is a registered trademark of Micro-Star Int'l Co.Ltd. All rights reserved.\\
    © 2024 NVIDIA, the NVIDIA logo, GeForce, SHIELD, and NVIDIA G-SYNC are trademarks and/or registered trademarks of NVIDIA Corporation in the U.S. and other countries.\\
    © Game Science Interactive Technology Co., Ltd. All Rights Reserved.
\end{enumerate}

\section{\nameref{cha:browsers-and-how-to-choose}}

\begin{enumerate}
  \item 浏览器份额变化图(\autoref{fig:Browser_market_share})\\
  原始数据来自 Statcounter(\url{https://gs.statcounter.com/press})。\\
  按 CC BY-SA 3.0 协议许可。\\
  我们将原始数据制成了曲线图。
\end{enumerate}

\section{\nameref{cha:mail-and-instant-messaging}}

\begin{enumerate}
  \item QQ 2010 主界面示意图(\autoref{fig:QQ2010})\\
    图片下载自《情感的容器 被寄托了的QQ2010视觉设计》(原始地址:\url{http://cdc.tencent.com/?p=2200};页面存档:\url{https://web.archive.org/web/20140816090252/http://cdc.tencent.com/?p=2200})。\\
    © 猕猴桃和 Tencent CDC。
\end{enumerate}

\section{\nameref{cha:bring-intelligence-to-machines}}

\begin{enumerate}
  \item NVIDIA V100 计算卡(\autoref{fig:V100})\\
    原图下载自网页「NVIDIA V100 | NVIDIA」(\url{https://www.nvidia.com/pt-br/data-center/v100/})。\\
    © 2024 NVIDIA, the NVIDIA logo, GeForce, SHIELD, and NVIDIA G-SYNC are trademarks and/or registered trademarks of NVIDIA Corporation in the U.S. and other countries.
\end{enumerate}

\section{\nameref{cha:program-and-arch}}

\begin{enumerate}
  \item 8086 处理器(\autoref{fig:8086})\\
    原图为 \textit{KL Intel D8086}(\url{https://commons.wikimedia.org/wiki/File:KL_Intel_D8086.jpg}),作者是 Konstantin Lanzet。\\
    按 CC BY-SA 3.0 协议许可。
  \item ARM 徽标(\autoref{fig:Arm_logo_2017})\\
    原图为 \textit{Arm logo 2017}(\url{https://commons.wikimedia.org/wiki/File:Arm_logo_2017.svg})。\\
    © ARM Holdings. ARM and the arm logo are trademarks of ARM Holdings.\\
    我们转换了图片的格式。
  \item ARM 610 核心照片(\autoref{fig:ARM610_die})\\
    原图为 \textit{GPS ARM610 die}(\url{https://commons.wikimedia.org/wiki/File:GPS_ARM610_die.JPG}),作者是 Pauli Rautakorpi。\\
    按 CC BY 3.0 协议许可。\\
    我们调整了图片的长宽比。
  \item 龙芯中科官网的 3A6000 介绍页(\autoref{fig:LS3A6K_webpage})\\
    截图自网页「龙芯 3A6000」(\url{https://www.loongson.cn/product/show?id=26})。\\
    © 龙芯中科技术股份有限公司。
\end{enumerate}
