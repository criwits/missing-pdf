\chapter{二次创作指南}
\label{cha:adapt-guide}

《你缺失的那门计算机课》网页版(下称《你缺计课》或「本作品」)是一套自由、开放、共享的网络教程。为了让《你缺计课》能更好地帮助更多的人,Hans 和 Windy(下称「我们」)选择「知识共享署名—非商业性使用—相同方式共享 4.0 协议国际版(CC BY-NC-SA 4.0)」(\url{https://creativecommons.org/licenses/by-nc-sa/4.0/deed.zh-hans})作为它的许可协议。任何人都可以在该协议允许的范围内,自由地对本作品进行传播、共享,并基于本作品进行二次创作(下文简称「二创」),如录制讲解视频、撰写扩展读物等。

\section{规矩}

\subsection{许可协议的继承性}

《你缺计课》使用的 CC BY-NC-SA 4.0 协议具有「继承性」。这意味着,任何基于本作品所二次创作的作品,都必须同样采用这一协议。二创作品应当在其合适的地方标注「本作品采用 CC BY-NC-SA 4.0 协议许可」等说明文字,并提供该许可协议的原文(\url{https://creativecommons.org/licenses/by-nc-sa/4.0/legalcode.zh-hans})或链接(\url{https://creativecommons.org/licenses/by/4.0/deed.zh-hans})。

\subsection{禁止用于商业用途}

我们禁止本作品被用于任何商业目的,即被用于索取金钱回报或商业优势。由于二创作品必须使用和本作品相同的许可协议,因此基于本作品的二创作品同样不可用于任何商业用途。例如:

\begin{itemize}
  \item 您可以将本作品下载整理为 Word 文档,{\color{MissingRed}但不得在互联网上售卖}。
  \item 您可以将本作品录制成有声书,{\color{MissingRed}但不得上传到可获得收益的平台}。
  \item 您可以制作基于本作品内容的讲解视频,{\color{MissingRed}但不得以之来开展以盈利为目的的培训,或在视频中通过广告、赞助等方式获得间接的收益}。
\end{itemize}

特别地,您需要注意您发布这些作品的平台是否存在隐性的商业行为,例如各类盈利性的「打赏」功能。同时,您不能将本作品及二创作品发布在付费订阅制的平台上。

\subsection{必须正确署名}

任何基于本作品的二创作品都必须在其合适的地方,说明其是基于原作品(《你缺失的那门计算机课》网页版)进行二次创作得来,并正确标注原作者信息(Hans Wan 和 Windy Deng),以及附上本作品的链接(\url{https://www.criwits.top/missing})。同时,还需要清楚说明原作品「以 CC BY-NC-SA 4.0 协议许可」并提供该许可协议的原文(\url{https://creativecommons.org/licenses/by-nc-sa/4.0/legalcode.zh-hans})或链接(\url{https://creativecommons.org/licenses/by/4.0/deed.zh-hans})。

以上要求是必须遵守的合规性要求。例如,一集基于本作品录制的讲解视频,可通过在片尾或片头展示如下的说明文字来满足这些要求。

\begin{quoting}
  本视频采用 CC BY-NC-SA 4.0 协议许可。\par
  您可以在 \url{https://creativecommons.org/licenses/by/4.0/deed.zh-hans} 查看该许可的详细信息。\par
  \phantom{六}\par
  原作:\par
  《你缺失的那门计算机课》网页版\par
  作者:Hans Wan 和 Windy Deng\par
  网址:\url{https://www.criwits.top/missing}\par
  以 CC BY-NC-SA 4.0 协议许可\par
\end{quoting}

我们保留一切合法维护自身权益的权利。任何违反这些「规矩」的二次创作者都将被我们以合适的方式公示,并有机会受到法律的制裁。

\section{建议}

\subsection{视觉形象与强调色}

《你缺失的那门计算机课》设计有自己的标志(logo,名为《疑?悟!》),二创作品可按需在引用本作品时使用。

\begin{figure}[htb!]
  \centering
  \includegraphics[width=.45\textwidth]{assets/missing_logo.pdf}
  \caption{疑?悟!}
  \label{fig:missing_logo}
\end{figure}

二创作品宜使用该标志中使用的蓝色或相近的颜色作为强调色。当需要多种不同强调色时,清新、明快的其他同风格色彩亦可使用。

\subsection{字体、字形与排版}

如果二创作品的载体允许自定义字体,我们建议二创作品对于主体部分的文字使用如下的字体组合:
\begin{itemize}
  \item 中文字体:思源黑体(\url{https://github.com/adobe-fonts/source-han-sans/})。
  \item 西文字体:Inter(\url{https://rsms.me/inter/})。
\end{itemize}

除少量用于强调、装饰或特定风格化内容以外,我们强烈建议不要使用「楷体」「仿宋」等书法字体,以及「得意黑」「Comic Sans」等美术字体。将这些字体大量用于作品主体部分(例如字幕、正文或说明性文字)会导致整体观感不佳,难以阅读。

在使用自定义字体时,我们建议创作者或排版人员使用符合中国内地国家标准的字形,避免错用港澳台以及日本、朝鲜、韩国等汉字使用地区的异体汉字,除非您面向的对象是上述地区的人员,在此情形下您应当将本作品译为对应的语言。

排版方面,对于中西文混排的句子,我们建议在所有西文(拉丁字母和阿拉伯数字)和汉字之间留有约 1/4 个汉字宽度的水平空隙。可以用半角空格来维持这一空隙。

如果二创作品的展示载体不能较好地区分前后 “弯引号”,我们建议在横排文本中使用「直角引号」来代替 “弯引号”,就像网页版《你缺计课》所做的那样。但请注意:在横排文本中使用直角引号不符合国家标准,不应在任何书面或印刷作品中使用。

我们建议清晰、合理地标记所有「在屏幕上显示的原文」,例如,使用方头括号「打开【此电脑】」,或在文字下方\CJKunderanyline*{0.5ex}{\color{MissingSkyBlue}\rule{2pt}{2.5ex}}{增加衬底}。

\subsection{演示环境}

除非有特殊需求,我们建议使用 Windows 11 家庭中文版作为您在二创作品中的演示平台。您在制作演示内容时,应当充分考虑读者所使用的环境的差异——例如,新安装的 Windows 系统通常在桌面上没有「此电脑」,并可能只有一个分区。

我们推荐您在条件允许的情况下,使用虚拟机来制作演示内容,以模拟出相对「纯净」的环境。

\subsection{自由和开源软件}

我们相信自由和开源软件(Free and Open Source Software,简称 FOSS)的力量,并始终支持自由和开源软件的发展。

我们非常鼓励您使用自由和开源软件制作二创作品。例如:使用 LibreOffice 撰写文字性的内容,使用 Inkscape、GIMP 和 Krita 绘制相关的图片,使用 Kdenlive 剪辑您的视频……如果您使用了 FOSS,您可以在二创作品的合适位置进行标注,这样能让更多的人了解到它们。

而对于本作品的软件篇,我们亦非常希望您推荐一批其他的优秀自由软件和开源软件。这有助于帮助它们获得更多的用户,并扩大自由软件运动精神的影响力。

\subsection{人工智能生成的内容(AIGC)}

我们强烈反对在《你缺计课》的二创作品中直接使用任何 AI 生成的内容,包括但不限于:文本内容、插图和配音。我们并非反对 AIGC——在创作过程中,您可以借助 AIGC 技术获取灵感和总结知识,但是\regcolor{除非有举例或叙述的必要,任何 AI 生成的内容都不应该直接进入《你缺计课》的二创作品}。